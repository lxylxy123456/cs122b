% template from http://nook.cs.ucdavis.edu/~koehl/Teaching/ECS20/homeworks.html
\documentclass[11pt]{article}
\usepackage{amsthm,amsmath, amssymb}
\setlength{\oddsidemargin}{0in}
\setlength{\evensidemargin}{0in}
\setlength{\textheight}{9.0in}
\setlength{\textwidth}{6.5in}
\setlength{\topmargin}{-0.5in}
\usepackage{graphicx}
\usepackage{listings}
\lstset{xleftmargin=.25in}  % https://tex.stackexchange.com/questions/10828/
\usepackage{multicol}
\usepackage{courier}
\usepackage{verbatim}
\lstset{basicstyle=\footnotesize\ttfamily,breaklines=true}
\usepackage{fancyhdr}
\usepackage{indentfirst}
\pagestyle{fancy}
\usepackage{hyperref}

% Due ???
\newcommand{\titleVar}{ECS 122B Spring 2019 - Randomized Algorithm Analysis}

\lhead{\titleVar{}} \chead{} \rhead{Eric Li}
\lfoot{} \cfoot{\thepage} \rfoot{}
\renewcommand{\headrulewidth}{0pt}
\renewcommand{\footrulewidth}{0pt}
\setlength\headsep{0.333in}
\title{\bf \titleVar{}}
\author{Eric Li}

\newcommand{\insertGraph}[3]{

  \centerline{\includegraphics[scale=#1]{#2}}

  \centerline{#3}

}

\newcommand{\insertTwoGraphs}[3]{

  \begin{multicols}{2}

  \insertGraph{#1}{#2.png}{#2}

  \insertGraph{#1}{#3.png}{#3}

  \end{multicols}

}

\begin{document}
\maketitle

\renewcommand\thesubsection{\arabic{section}.\Alph{subsection}}

This analysis adapts the idea in Introduction to Algorithms, third edition,
 chapter 7.4.2. % Page 181

Recall the selection algorithm. % Page 216
 The non-base case is first partitioning the array, then call RandomizedSelect
 on the partition of interest. 

The only time the algorithm compares is when partitioning the array, and similar
 to Lemma 7.1, we can see that the running time of RandomizedSelect is
 $O(n + X)$, where $X$ is the number of comparisons by the Partition algorithm.

Suppose we can sort the input array as $z_1, z_2, ..., z_n$, where $z_1$ is the
 smallest. Also, for simplicity, assume that $\forall i \neq j, z_i \neq z_j$. 

From similar reason for Quicksort, the time $z_i$ and $z_j$ are compared is at
 most $1$. Let $X_{ij} = I\{z_i \text{ is compared to } z_j\}$. There is
 $$X = \sum_{i=0}^n{\sum_{j=i+1}^n{X_{ij}}}$$

And since they are independent from each other, 
 $$E[X] = E\left[\sum_{i=0}^n{\sum_{j=i+1}^n{X_{ij}}}\right]
 		 = \sum_{i=0}^n{\sum_{j=i+1}^n{Pr\{z_i \text{ is compared to } z_j\}}}$$

So we need to compute $Pr\{z_i \text{ is compared to } z_j\}$. The difference
 to Quicksort is that this time we have another number $k$, which is the number
 of element in the list that we want to find. 

There are two cases. When $i \le k \le j$ (Note that there is $i < j$), $z_i$
 and $z_j$ get compared only when $z_i$ or $z_j$ is the first vertex chosen
 among $z_i, z_{i+1}, ..., z_j$. There are $j - i + 1$ vertices in this list,
 so $Pr\{z_i \text{ is compared to } z_j\} = \displaystyle\frac{2}{j-i+1}$. 

The second case is when $k$ is not between $i$ and $j$. Due to symmetricity, we
 only discuss the case where $k < i < j$ (case 2.a).
 Among $z_k, z_{k+1}, ..., z_i, z_{i+1}, ..., z_j$, if one of them other than
 $z_i$ and $z_j$ is chosen, $z_i$ and $z_j$ will not be compared. 
 The reason is that if a number between $k$ and $i$ is chosen, the partition
 $z_i$ and $z_j$ belong to will not be recursed on; if a number between $i$ and
 $j$ is chosen, $z_i$ and $z_j$ will belong to different partitions. Now, there
 are $j - k + 1$ numbers to choose from, and
 $Pr\{z_i \text{ is compared to } z_j\} = \displaystyle\frac{2}{j-k+1}$. 

Symmetrically, when $i < j < k$ (case 2.b),
 $Pr\{z_i \text{ is compared to } z_j\} = \displaystyle\frac{2}{k-i+1}$. 

Note that when $i = k$ or $j = k$, the first case is equivalent to the second
 case (2.a and 2.b, respectively). 

So $$
\begin{aligned}
E[X] & = \sum_{i=0}^n{\sum_{j=i+1}^n{Pr\{z_i \text{ cmp. to } z_j\}}} \\
	 & = \sum_{i=0}^{k-1}{\sum_{j=i+1}^n{Pr\{z_i \text{ cmp. to } z_j\}}} 
	   + \sum_{i=k}^n{\sum_{j=i+1}^n{Pr\{z_i \text{ cmp. to } z_j\}}} \\
	 & = \sum_{i=0}^{k-1}{\left(
	 			  \sum_{j=i+1}^k{Pr\{z_i \text{ cmp. to } z_j\}}
	 			+ \sum_{j=k+1}^n{Pr\{z_i \text{ cmp. to } z_j\}}
	 		\right)} 
	   + \sum_{i=k}^n{\sum_{j=i+1}^n{\displaystyle\frac{2}{j-k+1}}} \\
	 & = \sum_{i=0}^{k-1}{\left(
	 			  \sum_{j=i+1}^k{\displaystyle\frac{2}{k-i+1}}
	 			+ \sum_{j=k+1}^n{\displaystyle\frac{2}{j-i+1}}
	 		\right)} 
	   + \sum_{i=k}^n{\sum_{j=i+1}^n{\displaystyle\frac{2}{j-k+1}}} \\
	 & = \sum_{i=0}^{k-1}{\sum_{j=i+1}^k{\displaystyle\frac{2}{k-i+1}}}
	   + \sum_{i=0}^{k-1}{\sum_{j=k+1}^n{\displaystyle\frac{2}{j-i+1}}}
	   + \sum_{i=k}^n{\sum_{j=i+1}^n{\displaystyle\frac{2}{j-k+1}}} \\
	 & = \sum_{i=0}^{k-1}{\displaystyle\frac{2\,(k-i)}{k-i+1}}
	   + \sum_{i=0}^{k-1}{\sum_{j=k+1}^n{\displaystyle\frac{2}{j-i+1}}}
	   + \sum_{j=k+1}^n{\sum_{i=k}^{j-1}{\displaystyle\frac{2}{j-k+1}}} \\
	 & = \sum_{i=0}^{k-1}{\displaystyle\frac{2\,(k-i)}{k-i+1}}
	   + \sum_{i=0}^{k-1}{\sum_{j=k+1}^n{\displaystyle\frac{2}{j-i+1}}}
	   + \sum_{j=k+1}^n{\displaystyle\frac{2\,(j-k)}{j-k+1}} \\
	 & \le \sum_{i=0}^{k-1}{\displaystyle\frac{2\,(k-i+1)}{k-i+1}}
	   + \sum_{i=0}^{k-1}{\sum_{j=k+1}^n{\displaystyle\frac{2}{j-i+1}}}
	   + \sum_{j=k+1}^n{\displaystyle\frac{2\,(j-k+1)}{j-k+1}} \\
	 & = \sum_{i=0}^{k-1}{2}
	   + \sum_{i=0}^{k-1}{\sum_{j=k+1}^n{\displaystyle\frac{2}{j-i+1}}}
	   + \sum_{j=k+1}^n{2} \\
	 & = 2\,n
	   + \sum_{i=0}^{k-1}{\sum_{j=k+1}^n{\displaystyle\frac{2}{j-i+1}}}\\
	 & = 2\,n
	   + \sum_{i=0}^{k-1}{\sum_{j=1}^{n-k}{\displaystyle\frac{2}{j-k-i+1}}}\\
	 & = 2\,n
	   + \sum_{i=-k}^{-1}{\sum_{j=1}^{n-k}{\displaystyle\frac{2}{j-i+1}}}\\
	 & = 2\,n
	   + \sum_{i=1}^k{\sum_{j=1}^{n-k}{\displaystyle\frac{2}{j+i+1}}} \\
	 & = 2\,n
	   + 2\,\sum_{i=1}^k{\sum_{j=1}^{n-k}{\displaystyle\frac{1}{j+i+1}}}
\end{aligned}
$$

\begin{comment}
def check(n=10, k=4) :
	a = set()
	for i in range(k, n+1) :
		for j in range(i+1, n+1) :
			a.add((i, j))
	b = set()
	for j in range(k+1, n+1) :
		for i in range(k, j-1+1) :
			b.add((i, j))
	assert a == b
	return a, b
\end{comment}

Now, to prove that $X = O(n)$, we only need to prove that
 $$A = \sum_{i=1}^k{\sum_{j=1}^{n-k}{\displaystyle\frac{1}{j+i+1}}} = O(n)$$

\begin{comment}
Since the series does not always grow from $1$, we have to derive something
 similar to Equation A.10. % Page 1154 

(Note that $\lg$ has a base of $2$, as in Introduction to Algorithms). 

$$
\begin{aligned}
\sum_{k=m}^n{\frac{1}{k}}
& \le \sum_{i=\lfloor\lg{m}\rfloor}^{\lfloor\lg{n}\rfloor}{
			\sum_{j=0}^{2^i-1}{\frac{1}{2^i+j}}} \\
& \le \sum_{i=\lfloor\lg{m}\rfloor}^{\lfloor\lg{n}\rfloor}{
			\sum_{j=0}^{2^i-1}{\frac{1}{2^i}}} \\
& = \sum_{i=\lfloor\lg{m}\rfloor}^{\lfloor\lg{n}\rfloor}{1} \\
& \le \lg{n} - \lg{m} + 1
\end{aligned}
$$

So 
$$
\begin{aligned}
\sum_{i=1}^k{\sum_{j=1}^{n-k}{\displaystyle\frac{1}{j+i+1}}}
& \le \sum_{i=1}^k{\lg{(n-k+i+1)} - \lg{(i+2)} + 1} \\
& \le k + \sum_{i=1}^k{\lg{(n-k+i+1)} - \lg{(i+2)}} \\
& = k + \sum_{i=1}^k{\lg{(n-k+i+1)}} - \sum_{i=1}^k{\lg{(i+2)}} \\
& = k + \sum_{i=n-k+2}^{n+1}{\lg{i}} - \sum_{i=3}^{k+2}{\lg{i}} \\
\end{aligned}
$$

We have $k = O(n)$, and let
$$A = \sum_{i=n-k+2}^{n+1}{\lg{i}} - \sum_{i=3}^{k+2}{\lg{i}}$$

We need to prove that $A = O(n)$. We can consider whether the two summation
 have overlap items and split into two cases

Case 1: when $n - k > k$ (i.e. $k < n/2$), let $k' = k$, so
 $$A = \sum_{i=n-k'+2}^{n+1}{\lg{i}} - \sum_{i=3}^{k'+2}{\lg{i}}$$

Case 2: when $n - k \le k$ (i.e. $k \ge n/2$), let $k' = n - k$
 $$
 \begin{aligned}
 A & = \sum_{i=n-n+k'+2}^{n+1}{\lg{i}} - \sum_{i=3}^{n-k'+2}{\lg{i}} \\
	& = \sum_{i=k'+2}^{n+1}{\lg{i}} - \sum_{i=3}^{n-k'+2}{\lg{i}} \\
	& = \sum_{i=k'+2}^{n+1}{\lg{i}} - \sum_{i=k'+2}^{n-k'+2}{\lg{i}}
	 + \sum_{i=k'+2}^{n-k'+2}{\lg{i}} - \sum_{i=3}^{n-k'+2}{\lg{i}} \\
	& = \sum_{i=n-k'+2}^{n+1}{\lg{i}} - \sum_{i=3}^{k'+2}{\lg{i}} \\
 \end{aligned}
 $$

So we can see that whichever the case is, $\exists k' \le n/2$ such that
 $$A = \sum_{i=n-k'+2}^{n+1}{\lg{i}} - \sum_{i=3}^{k'+2}{\lg{i}}$$

Now we can drop the lower term

$$
\begin{aligned}
A & = \sum_{i=n-k'+2}^{n+1}{\lg{i}} - \sum_{i=3}^{k'+2}{\lg{i}} \\
 & \le \sum_{i=n-k'+2}^{n+1}{\lg{i}} \\
 & \le \sum_{i=n-\lfloor\frac{n}2\rfloor+2}^{n+1}{\lg{i}} \\
 & \le \sum_{i=\lceil\frac{n}2\rceil}^{n+1}{\lg{i}} \\
 0/0
\end{aligned}
$$
\end{comment}

Due to symmetricity, we can assume that $k \le n/2$, which implies $k \le n-k$. 

The summation becomes
$$
\begin{aligned}
A & = 
% \frac{1}{3} + \frac{2}{4} + \frac{3}{5} + ... + \frac{k}{k+2} + \frac{k}{k+3}
% + \frac{k}{k+4} + ... + \frac{k}{n-k+2} + \frac{k-1}{k-k+3} + ...
% + \frac{1}{n+1} \\
	\sum_{i=1}^k{\sum_{j=1}^{n-k}{\displaystyle\frac{1}{j+i+1}}} \\
& = \sum_{i=1}^k{\sum_{j=1+i}^{n-k+i}{\displaystyle\frac{1}{j+1}}} \\
& = \sum_{i=1}^k{\left(
		\sum_{j=1+i}^{k}{\displaystyle\frac{1}{j+1}} + 
		\sum_{j=k}^{n-k+1}{\displaystyle\frac{1}{j+1}} + 
		\sum_{j=n-k+1}^{n-k+i}{\displaystyle\frac{1}{j+1}}
	\right)} \\
& = \sum_{i=1}^k{\sum_{j=1+i}^{k}{\displaystyle\frac{1}{j+1}}} + 
	\sum_{i=1}^k{\sum_{j=k}^{n-k+1}{\displaystyle\frac{1}{j+1}}} + 
	\sum_{i=1}^k{\sum_{j=n-k+1}^{n-k+i}{\displaystyle\frac{1}{j+1}}} \\
& = \sum_{j=2}^k{\sum_{i=1}^{j-1}{\displaystyle\frac{1}{j+1}}} + 
	\sum_{j=k}^{n-k+1}{\displaystyle\frac{k}{j+1}} + 
	\sum_{j=n-k+1}^{n-k+k}{\sum_{i=j-n+k}^{k}{\displaystyle\frac{1}{j+1}}} \\
& = \sum_{j=2}^k{\displaystyle\frac{j-1}{j+1}} + 
	\sum_{j=k}^{n-k+1}{\displaystyle\frac{k}{j+1}} + 
	\sum_{j=n-k+1}^{n}{\displaystyle\frac{n-j+1}{j+1}} \\
& \le \sum_{j=2}^k{1} + 
	\sum_{j=k}^{n-k+1}{1} + 
	\sum_{j=n-k+1}^{n}{1} \\
& \le \sum_{j=2}^n{1} \\
& \le n
\end{aligned}
$$

So $A = O(n)$. 

So $E[X] = O(n)$. 

So the expected running time of RandomizedSelect is $O(n + X) = O(n)$. 

\end{document}

